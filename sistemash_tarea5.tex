\documentclass{article}

\usepackage[utf8]{inputenc}
\usepackage{graphicx}   % Para imágenes.
\usepackage{multicol}
\usepackage{amsmath}
\usepackage{dashrule}
\usepackage{geometry}
\usepackage[spanish, mexico]{babel}
\usepackage{subcaption}
\usepackage[svgnames]{xcolor}
\usepackage{tcolorbox}
\usepackage[table,xcdraw]{xcolor}
\usepackage{fancyhdr}
\usepackage{enumitem}
\usepackage{siunitx}
\usepackage[export]{adjustbox}
\usepackage{multirow}


\usepackage[normalem]{ulem}
\useunder{\uline}{\ul}{}

\definecolor{gray(x11gray)}{rgb}{0.75, 0.75, 0.75}
\definecolor{outerspace}{rgb}{0.25, 0.29, 0.3}
\definecolor{pastelgreen}{rgb}{0.47, 0.87, 0.47}
\definecolor{lincolngreen}{rgb}{0.11, 0.35, 0.02}

\geometry{
    a4paper,
    tmargin = 1.7 cm,
    bmargin = 1.7cm,
    lmargin = 1.5cm,
    rmargin = 1.5cm
}



\pagestyle{fancy}
\fancyhf{}
\cfoot{ \thepage  \hspace{0.5pt}\hspace{0.5pt}}
\lhead{Tarea Examen}

\begin{document}

\thispagestyle{plain}


\hrule
\begin{center}
    {\Large \textbf{UNIVERSIDAD NACIONAL AUTÓNOMA DE MÉXICO}}
    \vspace{10pt}

    {\Large{{Sistemas hibridos en biomedicina}}}
    
    
    \vspace{10pt}

    \hrule

    \vspace{20pt}


    {\Huge \textbf{Tarea 5: Radiofármacos}}\\
\end{center}

\hdashrule{\linewidth}{1pt}{1mm}

\begin{flushright}
    {\small Medel Garduño Diego} 
\end{flushright}

\section{Radiofármacos utilizados en PET/medicina nuclear}

\subsection{F-FDG: 2 - [18F] Fluoro - 2 - desoxi - D - glucosa }


Este radiofarmaco es uno de los más utilizados en la medicina nuclear, es un de la glucosa, es decir, sigue la misma ruta metabolica y contiene ${}^{18} F$ un radioisotopo emisor de particulas $\beta + $. Su principal uso esta destinado a la oncología, puesto que las celulas que consumen mayor cantidad de glucosa son los tumores, entonces una mayor captación de glucosa, implica una mayor captacion de F-FDG, lo que a su vez es sinomino de una tasa elevada de crecimiento celular. 


\subsection{F-FCH: Flurometil - dimetil - 2 - hidroxietil - amonio}

Este radiofarmaco F-FCH está relacionado con la membrana celular,  las celulas tumorales se caracterizan por presentar un aumento en la biosintesis de membrana celular, por lo que la captación de este radiofarmaco esta relacionado con la proliferación celular. La F-FCH se utiliza ante indicios de la existencia de tumores cerebrales y de hepatocarcinomas.


\subsection{${}^{18} F$ - FDOPA: L-6[18F] fluoro - 3,4 - dihidroxifenilalanina}

La L-DOPA es un aminoacido precursor de la dopamina, la adrenalina y la noradranalina, de manera original este radiofarmaco tuvo implicaciones en el estudio de la integridad de la via dopaminérgica del estriado en pacienctes con enfermedad de Parkinson. Tambien tiene usos en el diagnostico y la localización de insulinomas en casos de hiperinsulinismo en lactantes y niños


\section*{Referencias}

\begin{enumerate}
    \item [[\textcolor{blue}{1}]] Olivas Arroyo, C. (2016). Radiofármacos utilizados en la tomografía por emisión de positrones: presente y perspectivas de futuro [Radiopharmaceuticals in positron emission tomography: Present situation and future perspectives]. Radiología, 58(6), 433-442. https://doi.org/10.1016/j.rx.2016.07.002
    
\end{enumerate}







\end{document}